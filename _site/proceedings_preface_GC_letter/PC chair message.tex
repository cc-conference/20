\documentclass[12pt,letterpaper]{article}

% encoding and fonts first
\usepackage[utf8]{inputenc}
\usepackage[T1]{fontenc}
\usepackage{microtype}
\usepackage[tt=false, type1=true]{libertine}
\sloppy

\usepackage{geometry}
%% For preface:
\geometry{textwidth=14cm,textheight=20cm}
%% For committee listings and sponsor pages:
%\geometry{textwidth=18cm,textheight=23.5cm}

%% For format `acmsmall'
%\geometry{twoside=true,
%          includeheadfoot, head=13pt, foot=2pc,
%          paperwidth=6.75in, paperheight=10in,
%          top=58pt, bottom=44pt, inner=46pt, outer=46pt,
%          marginparwidth=2pc,heightrounded
%         }

%% For IEEE conferences and workshops, please uncomment the following line to use Times.
%\usepackage{times}


%-------------------------------------------------------------------------
\begin{document}

\title{\sffamily\bfseries Welcome from the Program Chair}
\date{}

\maketitle
\thispagestyle{empty}
\pagestyle{empty}

Welcome to the 28th International Conference on Compiler Construction (CC), held in Washington, DC, February 16--17 2019. CC continues its focus on processing programs in the most general sense: analyzing, transforming or executing input that describes how a system operates. I think our program demonstrates the breadth of work that CC welcomes, ranging from classic code generation concerns like vectorization to new APIs to help speed up web applications.

Submissions were gathered and managed through HotCRP, and I would like to thank Eddie Kohler for his assistance in making sure the process went smoothly. I would like to thank the program committee and external reviewers for their hard work in reviewing the submissions. Each paper was reviewed by at least three PC members.  This year, we used an asynchronous, fully-online process to make our acceptance decisions: papers were discussed using HotCRP's messaging tools, and decisions were made by consensus. The review process was double-blind from start to finish: accepted papers were unblinded only once the list of accepted papers was posted, and rejected papers were not unblinded.
Overall, we received 45 submissions, of which 17 were accepted, for an acceptance rate of 38\%. 

I asked the Program Committee to nominate papers that were worthy of a best paper award, then asked a subset of the PC to review the four papers that rose to the top, from which we selected our best paper: TK TK TK

Our program is divided into five sessions across one and a half days. We thank all of the volunteers who are chairing these sessions. We also would like to thank Saman Amarasinghe for graciously agreeing to present the CC Keynote on the ``Sparse Tensor Algebra Compiler.''

Finally, I would like to thank the General Chair (J. Nelson Amaral), last year's program chair (Jingling Xue), and the members of the steering committee (especially Albert Cohen and Sebastian Hack) for their advice and help during the process of putting together this year's conference.

%
%\textbf{ACM}: The titles (Biolinum, sans-serif, 18 point, bold), the headings (Biolinum, sans-serif, 14 point, bold), 
%and the text (Libertine, serif, 12 point) fill the full width of the page – one column.
%\textbf{IEEE}: The titles (Helvetica/Arial, 18 point, bold), the headings (Helvetica/Arial, 14 point, bold), 
%and the text (Times New Roman, 12 point) fill the full width of the page – one column.
%
%No classifiers and no copyright block at the bottom of the first page are needed. No transfer of copyright or permission release is necessary.
%
%Page numbers should be omitted.
%
%Text and URLs should be in black, not in blue/other color, and not underlined.
%
%If the submission system asks for an abstract, please copy the first paragraph of the welcome message (or whatever is appropriate) as abstract. This will be shown as description in HTML navigation structures.
%
%The deadline for submission of this preface is one week after the camera-ready paper deadline for the papers.
%
\bigskip
\noindent
West Lafayette, IN       \hfill Milind Kulkarni\\
February 2019 \hfill PC Chair

\end{document}
